\documentclass{beamer}

\mode<presentation>
{
  \usetheme{CambridgeUS}      % or try Darmstadt, Madrid, Warsaw, ...
  \usecolortheme{default} % or try albatross, beaver, crane, ...
  \usefonttheme{default}  % or try serif, structurebold, ...
  \setbeamertemplate{navigation symbols}{}
  \setbeamertemplate{caption}[numbered]
  \setbeamertemplate{bibliography item}[text]
  \setbeamertemplate{footline}{}
} 

\usepackage[english]{babel}
\usepackage[utf8x]{inputenc}
\usepackage[document]{ragged2e}

\usepackage{color}

\titlegraphic{\includegraphics[scale=0.24]{../Figures/cover.jpeg}}
\title[The Craft of Research]{The Craft of Research}
\author{Bilgin AKSOY}
\institute{Informatics Institute}
\logo{\includegraphics[scale=0.05]{../Figures/mmilogo.png}}

\begin{document}

	\begin{frame}
	  \titlepage
	\end{frame}

% Problem definitons
\section{Introduction}
	\begin{frame}
	\begin{block}{Authors}
	\end{block}
		\begin{itemize}
			\item Wayne C. Booth (Prof. in English Teaching-Literature)
			\item Gregory G. Colomb (Prof. in English Teaching-Literature)
			\item Joseph M. Williams (Prof. in English Teaching-Literature)
			\item Joseph Bizup (Assoc. Prof. in English)
			\item William T. Fitzgerald (Assoc. Prof. in English)
		\end{itemize}			 
	\end{frame}
	\begin{frame}{Information}
		\begin{itemize}
			\item This is the fourth edition(1995, 2003, 2008, 2016).
			\item The publisher is The University of Chicago. 
			\item Subjects: \href{http://library.metu.edu.tr/search~S4/?searchtype=d&searcharg=research+methodolgy&searchscope=4&sortdropdown=-&SORT=DZ&extended=0&SUBMIT=Search&searchlimits=&searchorigarg=Xreaerch+methodolgy}{Research--Methodology | Technical Writing}
			\item 316 pages
			\item It is available in METU Book-store.
		\end{itemize}
	\end{frame}
	
	\setbeamercovered{transparent}
	\begin{frame}{Who should read this book? And What to Read?}
		\begin{itemize}
			\item All researchers: not only the students but also those in business and government who do and report research on any topic. The book covers :
			\begin{itemize}
				\onslide<1-> \item Topic-Questions-Problem-Sources
				\onslide<2-> \item Argument-Claim-Reasons-Evidence-Acknowledgements-Responses-Warrants
				\onslide<3-> \item Writing (Planning-Organizing-Styling)
			\end{itemize}
		\end{itemize}	
	\end{frame}
	
\section{Topic-Questions-Problem-Sources}	
	%next slide
	\begin{frame}
		\begin{block}{Topic}
		\end{block}
		\begin{itemize}
			\onslide<1-> \item Start listing your interests.
			\onslide<2-> \item Select one or two of them.
			\onslide<3-> \item Search in a library, and online databases (CQ Researcher, Academic Search Premier, Google Scholar or domain specific databeses like IEEE etc.) Don't use Wikipedia.
			\onslide<4-> \item After selecting one broad topic, try to narrow it. 
		\end{itemize}
	\end{frame}
	%next slide
	\begin{frame}
		\begin{block}{Questions}
		\end{block}
		\begin{itemize}
			\onslide<1-> \item Once selecting the topic, ask questions to engage your best critical thinking. (Who, What, When, Where, \textbf{How, and Why})
			\onslide<2-> \item Ask yourself So What? (Is there any cost/loss if you don't answer the question?) 
		\end{itemize}
	\end{frame}
	%next slide
	\begin{frame}
		\begin{block}{Problem}
		\end{block}
		\begin{itemize}
			\onslide<1-> \item After questioning the topic, try to find the significance from your readers' point of view. 
			\onslide<2-> \item And find a good research problem. (Set realistic goals.) 
		\end{itemize}
	\end{frame}
	%next slide
	\begin{frame}
		\begin{block}{Sources}
		\end{block}
		\begin{itemize}
			\onslide<1-> \item \textbf{Primary Sources:} Original materials, Historical records, raw data. (Exp. if you cite Bayes Theorem  than cite the original document written by Thomas Bayes.)
			\onslide<2-> \item \textbf{Secondary Sources:} Books, articles, or reports based on primary resources and written for scholar or professional purposes.(Only cite secondary sources if accessing a primary source is impossible.) 
			\onslide<3-> \item \textbf{Tertiary Sources:} Books, articles, or reports based on primary resources and written for general reader. 
		\end{itemize}
	\end{frame}
	
	%next slide
	\begin{frame}
		\begin{block}{Sources}
		\end{block}
		\begin{itemize}
			\onslide<1-> \item \textbf{Primary Sources:} Original materials, Historical records, raw data. (Exp. if you cite Bayes Theorem  than cite the original document written by Thomas Bayes.)
			\onslide<2-> \item \textbf{Secondary Sources:} Books, articles, or reports based on primary resources and written for scholar or professional purposes.(Only cite secondary sources if accessing a primary source is impossible.) 
			\onslide<3-> \item \textbf{Tertiary Sources:} Books, articles, or reports based on primary resources and written for general reader. 
		\end{itemize}
	\end{frame}
	%next slide
	\begin{frame}
		\begin{block}{Tip (Source)}
		\end{block}
		\begin{itemize}
			\onslide<1-> \item \textbf{Call Number (LCCN):} In book's details page.(LCCN: 2016000143)
			\onslide<2-> \item \textbf{Subject Headings (LCSH):} In book's details page. \href{http://library.metu.edu.tr/search~S4/?searchtype=d&searcharg=research+methodolgy&searchscope=4&sortdropdown=-&SORT=DZ&extended=0&SUBMIT=Search&searchlimits=&searchorigarg=Xreaerch+methodolgy}{LCSH: Research--Methodology | Technical Writing } 
			\onslide<3-> \item \textbf{Citation Index:} Helps finding a proper source.
			\onslide<4-> \item \textbf{Bibliographical Information :} You should give all bibliographical information using the proper citing style. Using reference management system (Mendeley etc.) will help doing it easier. 
			\onslide<4-> \item \textbf{Note-Taking :} Note taking is an important part of the research project. A note-taking application will be helpful. (OneNote, EverNote, and EndNote etc.)
		\end{itemize}
	\end{frame}
\section{Argument-Claim-Reasons-Evidence-Acknowledgements-Responses-Warrants}	
	%next slide
	\begin{frame}
		\begin{block}{Argument}
		\end{block}
		\begin{itemize}
			\onslide<1-> \item \textbf{Call Number (LCCN):} 
		\end{itemize}
	\end{frame}
	%next slide
	\begin{frame}
	\begin{table}[H] % H stands for here not anywhere else
			\centering	
			\caption[The Time-Consuming Of Serial And Parallel Bilinear Spatial Interpolation Of Different Grid Size]{\justifying The Time-Consuming Of Serial And Parallel Bilinear Spatial Interpolation Of Different Grid Size}
			\label{tab:table_1}
			\begin{tabular}{l c c }
				Grid Size & CPU (sec) & GPU (sec)\\ \hline
				40x40 & 1.70 & 0.11 \\  
				100x100 & 10.34 & 0.37 \\ 
				200x200 & 41.43 & 1.21 \\ 
				300x300 & 93.13 & 2.36 \\ 
				500x500 & 240.02 & 5.70 \\ 
				800x800 & 630.92 & 14.40 \\ 
				1200x1200 & 1432.33 &  32.41 \\ \hline
			\end{tabular}
		\end{table}
	\end{frame}
	
%References
\section{References}
	%next slide
	\begin{block}{References}
	\end{block}
%	\bibliography{./References/ref.bib}
%	\bibliographystyle{ieeetr}

\end{document}
